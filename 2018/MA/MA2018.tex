\documentclass[11pt]{article}
\usepackage{amsmath}
\usepackage{amssymb}
\usepackage{graphicx}
\usepackage{tabularx}
\usepackage{fancyhdr}
\usepackage{lastpage}

% Page layout
\usepackage[top=1in, bottom=1in, left=1in, right=1in]{geometry}

% Header and footer
\pagestyle{fancy}
\fancyhf{}
\rfoot{Page \thepage}
\renewcommand{\headrulewidth}{0pt}

% Modified Question command with left-aligned number
\newcommand{\questiona}[2]{
    \noindent\textbf{Q#2.} #1 \hfill \textbf{[1 Mark]}
}

\newcommand{\questionb}[2]{
    \noindent\textbf{Q#2.} #1 \hfill \textbf{[2 Marks]}
}

\begin{document}

% Title section with horizontal line
\begin{center}
    \Large\textbf{GATE 2018 - Mathematics (MA)} \\
    \large\textbf{General Aptitude and Technical Questions} \\
    \rule{\textwidth}{0.5pt} % Horizontal line below heading
\end{center}

\vspace{0.5cm}

% General Aptitude Section
\section*{General Aptitude}

\questiona{The dress \_\_\_\_\_ her so well that they all immediately \_\_\_\_\_ her on her appearance.}{1}
\begin{enumerate}
    \item[(A)] complemented, complemented  
    \item[(B)] complimented, complemented  
    \item[(C)] complimented, complimented  
    \item[(D)] complemented, complimented  
\end{enumerate}
\vspace{0.5cm}

\questiona{The judge’s standing in the legal community, though shaken by false allegations of wrongdoing, remained \_\_\_\_\_.}{2}
\begin{enumerate}
    \item[(A)] undiminished  
    \item[(B)] damaged  
    \item[(C)] illegal  
    \item[(D)] uncertain  
\end{enumerate}
\vspace{0.5cm}

\questiona{Find the missing group of letters in the following series: BC, FGH, LMNO, \_\_\_\_\_.}{3}
\begin{enumerate}
    \item[(A)] UVWXY  
    \item[(B)] TUVWX  
    \item[(C)] STUVW  
    \item[(D)] RSTUV  
\end{enumerate}
\vspace{0.5cm}

\questiona{The perimeters of a circle, a square and an equilateral triangle are equal. Which one of the following statements is true?}{4}
\begin{enumerate}
    \item[(A)] The circle has the largest area.  
    \item[(B)] The square has the largest area.  
    \item[(C)] The equilateral triangle has the largest area.  
    \item[(D)] All the three shapes have the same area.  
\end{enumerate}
\vspace{0.5cm}

\questiona{The value of the expression \[ \frac{1}{1+\log_u vw} + \frac{1}{1+\log_v wu} + \frac{1}{1+\log_w uv} \] is \_\_\_\_\_.}{5}
\begin{enumerate}
    \item[(A)] -1  
    \item[(B)] 0  
    \item[(C)] 1  
    \item[(D)] 3  
\end{enumerate}
\vspace{0.5cm}

\questionb{Forty students watched films A, B and C over a week. Each student watched either only one film or all three. Thirteen students watched film A, sixteen students watched film B and nineteen students watched film C. How many students watched all three films?}{6}
\begin{enumerate}
    \item[(A)] 0  
    \item[(B)] 2  
    \item[(C)] 4  
    \item[(D)] 8  
\end{enumerate}
\vspace{0.5cm}

\questionb{A wire would enclose an area of 1936 m\(^2\), if it is bent into a square. The wire is cut into two pieces. The longer piece is thrice as long as the shorter piece. The long and the short pieces are bent into a square and a circle, respectively. Which of the following choices is closest to the sum of the areas enclosed by the two pieces in square meters?}{7}
\begin{enumerate}
    \item[(A)] 1096  
    \item[(B)] 1111  
    \item[(C)] 1243  
    \item[(D)] 2486  
\end{enumerate}
\vspace{0.5cm}

\questionb{A contract is to be completed in 52 days and 125 identical robots were employed, each operational for 7 hours a day. After 39 days, five-seventh of the work was completed. How many additional robots would be required to complete the work on time, if each robot is now operational for 8 hours a day?}{8}
\begin{enumerate}
    \item[(A)] 50  
    \item[(B)] 89  
    \item[(C)] 146  
    \item[(D)] 175  
\end{enumerate}
\vspace{0.5cm}

\questionb{A house has a number which needs to be identified. The following three statements are given that can help in identifying the house number.\\
(i) If the house number is a multiple of 3, then it is a number from 50 to 59.\\
(ii) If the house number is NOT a multiple of 4, then it is a number from 60 to 69.\\
(iii) If the house number is NOT a multiple of 6, then it is a number from 70 to 79.\\
What is the house number?}{9}
\begin{enumerate}
    \item[(A)] 54  
    \item[(B)] 65  
    \item[(C)] 66  
    \item[(D)] 76  
\end{enumerate}
\vspace{0.5cm}

\questionb{An unbiased coin is tossed six times in a row and four different such trials are conducted. One trial implies six tosses of the coin. If H stands for head and T stands for tail, the following are the observations from the four trials: \\
(1) HTHTHT \quad (2) TTHHHT \quad (3) HTTHHT \quad (4) HHHT\_\_ \_\_ \\
Which statement describing the last two coin tosses of the fourth trial has the highest probability of being correct?}{10}
\begin{enumerate}
    \item[(A)] Two T will occur.  
    \item[(B)] One H and one T will occur.  
    \item[(C)] Two H will occur.  
    \item[(D)] One H will be followed by one T.  
\end{enumerate}
\vspace{0.5cm}

\section*{Technical Section}

\questiona{The principal value of \( (-1)^{\frac{-2i}{\pi}} \) is}{1}
\begin{enumerate}
    \item[(A)] \( e^2 \)
    \item[(B)] \( e^{2i} \)
    \item[(C)] \( e^{-2i} \)
    \item[(D)] \( e^{-2} \)
\end{enumerate}
\vspace{0.5cm}

\questiona{Let \( f : \mathbb{C} \to \mathbb{C} \) be an entire function with \( f(0) = 1 \), \( f(1) = 2 \), and \( f'(0) = 0 \). If there exists \( M > 0 \) such that \( |f''(z)| \leq M \) for all \( z \in \mathbb{C} \), then \( f(2) = \)}{2}
\begin{enumerate}
    \item[(A)] 2
    \item[(B)] 5
    \item[(C)] \( 2 + 5i \)
    \item[(D)] \( 5 + 2i \)
\end{enumerate}
\vspace{0.5cm}

\questiona{In the Laurent series expansion of \( f(z) = \frac{1}{z(z - 1)} \) valid for \( |z - 1| > 1 \), the coefficient of \( \frac{1}{z - 1} \) is}{3}
\begin{enumerate}
    \item[(A)] -2
    \item[(B)] -1
    \item[(C)] 0
    \item[(D)] 1
\end{enumerate}
\vspace{0.5cm}

\questiona{Let \( X \) and \( Y \) be metric spaces, and let \( f : X \to Y \) be a continuous map. For any subset \( S \) of \( X \), which one of the following statements is true?}{4}
\begin{enumerate}
    \item[(A)] If \( S \) is open, then \( f(S) \) is open
    \item[(B)] If \( S \) is connected, then \( f(S) \) is connected
    \item[(C)] If \( S \) is closed, then \( f(S) \) is closed
    \item[(D)] If \( S \) is bounded, then \( f(S) \) is bounded
\end{enumerate}
\vspace{0.5cm}

\questiona{The general solution of the differential equation \( x y' = y + \sqrt{x^2 + y^2} \) for \( x > 0 \) is given by (with an arbitrary positive constant \( k \))}{5}
\begin{enumerate}
    \item[(A)] \( k y^2 = x + \sqrt{x^2 + y^2} \)
    \item[(B)] \( k x^2 = x + \sqrt{x^2 + y^2} \)
    \item[(C)] \( k x^2 = y + \sqrt{x^2 + y^2} \)
    \item[(D)] \( k y^2 = y + \sqrt{x^2 + y^2} \)
\end{enumerate}
\vspace{0.5cm}

\questiona{Let \( p_n(x) \) be the polynomial solution of the differential equation \( \frac{d}{dx} \left[(1 - x^2) y' \right] + n(n + 1)y = 0 \) with \( p_n(1) = 1 \) for \( n = 1, 2, 3, \ldots \). If \( \frac{d}{dx}[p_{n+2}(x) - p_n(x)] = \alpha_n p_{n+1}(x) \), then \( \alpha_n \) is}{6}
\begin{enumerate}
    \item[(A)] \( 2n \)
    \item[(B)] \( 2n + 1 \)
    \item[(C)] \( 2n + 2 \)
    \item[(D)] \( 2n + 3 \)
\end{enumerate}
\vspace{0.5cm}

\questiona{In the permutation group \( S_6 \), the number of elements of order 8 is}{7}
\begin{enumerate}
    \item[(A)] 0
    \item[(B)] 1
    \item[(C)] 2
    \item[(D)] 4
\end{enumerate}
\vspace{0.5cm}

\questiona{Let \( R \) be a commutative ring with 1 (unity) which is not a field. Let \( I \subset R \) be a proper ideal such that every element of \( R \setminus I \) is invertible in \( R \). Then the number of maximal ideals of \( R \) is}{8}
\begin{enumerate}
    \item[(A)] 1
    \item[(B)] 2
    \item[(C)] 3
    \item[(D)] infinite
\end{enumerate}
\vspace{0.5cm}

\questiona{Let \( f : \mathbb{R} \to \mathbb{R} \) be a twice continuously differentiable function. The order of convergence of the secant method for finding a root of the equation \( f(x) = 0 \) is}{9}
\begin{enumerate}
    \item[(A)] \( \frac{1 + \sqrt{5}}{2} \)
    \item[(B)] \( \frac{2}{1 + \sqrt{5}} \)
    \item[(C)] \( \frac{1 + \sqrt{5}}{3} \)
    \item[(D)] \( \frac{3}{1 + \sqrt{5}} \)
\end{enumerate}
\vspace{0.5cm}

\questiona{The Cauchy problem \( u u_x + y u_y = x \) with \( u(x,1) = 2x \), when solved using its characteristic equations with an independent variable \( t \), is found to admit of a solution in the form \( x = \frac{3}{2} s e^t - \frac{1}{2} s e^{-t}, y = e^t, u = f(s, t) \). Then \( f(s, t) = \)}{10}
\begin{enumerate}
    \item[(A)] \( \frac{3}{2} s e^t + \frac{1}{2} s e^{-t} \)
    \item[(B)] \( \frac{1}{2} s e^t + \frac{3}{2} s e^{-t} \)
    \item[(C)] \( \frac{1}{2} s e^t - \frac{3}{2} s e^{-t} \)
    \item[(D)] \( \frac{3}{2} s e^t - \frac{1}{2} s e^{-t} \)
\end{enumerate}
\vspace{0.5cm}

\questiona{An urn contains four balls, each ball having equal probability of being white or black. Three black balls are added to the urn. The probability that five balls in the urn are black is}{11}
\begin{enumerate}
    \item[(A)] \( \frac{2}{7} \)
    \item[(B)] \( \frac{3}{8} \)
    \item[(C)] \( \frac{1}{2} \)
    \item[(D)] \( \frac{5}{7} \)
\end{enumerate}
\vspace{0.5cm}

\questiona{For a linear programming problem, which one of the following statements is FALSE?}{12}
\begin{enumerate}
    \item[(A)] If a constraint is an equality, then the corresponding dual variable is unrestricted in sign
    \item[(B)] Both primal and its dual can be infeasible
    \item[(C)] If primal is unbounded, then its dual is infeasible
    \item[(D)] Even if both primal and dual are feasible, the optimal values of the primal and the dual can differ
\end{enumerate}
\vspace{0.5cm}

\questiona{Let \( A = \begin{bmatrix} a & 2f & 0 \\ 2f & b & 3f \\ 0 & 3f & c \end{bmatrix} \), where \( a, b, c, f \in \mathbb{R} \) and \( f \neq 0 \). The geometric multiplicity of the largest eigenvalue of \( A \) equals}{13}
\vspace{0.5cm}

\questiona{Consider the subspaces \( W_1 = \{(x_1, x_2, x_3) \in \mathbb{R}^3 : x_1 = x_2 + 2x_3\} \) and \( W_2 = \{(x_1, x_2, x_3) \in \mathbb{R}^3 : x_1 = 3x_2 + 2x_3\} \) of \( \mathbb{R}^3 \). Then the dimension of \( W_1 + W_2 \) equals}{14}
\vspace{0.5cm}

\questiona{Let \( V \) be the real vector space of all polynomials of degree less than or equal to 2 with real coefficients. Let \( T : V \to V \) be the linear transformation given by \( T(p) = 2p + p' \) for \( p \in V \), where \( p' \) is the derivative of \( p \). Then the number of nonzero entries in the Jordan canonical form of a matrix of \( T \) equals}{15}
\vspace{0.5cm}

\questiona{Let \( I = [2, 3) \), \( J \) be the set of all rational numbers in the interval [4, 6], \( K \) be the Cantor (ternary) set, and let \( L = \{7 + x : x \in K\} \). Then the Lebesgue measure of the set \( I \cup J \cup L \) equals}{16}
\vspace{0.5cm}

\questiona{Let \( u(x, y, z) = x^2 - 2y + 4z^2 \) for \( (x, y, z) \in \mathbb{R}^3 \). Then the directional derivative of \( u \) in the direction \( \frac{3}{5} \hat{i} - \frac{4}{5} \hat{k} \) at the point \( (5, 1, 0) \) is}{17}
\vspace{0.5cm}

\questiona{If the Laplace transform of \( y(t) \) is given by \( Y(s) = \mathcal{L}(y(t)) = \frac{5}{2(s - 1)} - \frac{2}{s - 2} + \frac{1}{2(s - 3)} \), then \( y(0) + y'(0) = \)}{18}
\vspace{0.5cm}

\questiona{The number of regular singular points of the differential equation \[ [(x - 1)^2 \sin x] y'' + [\cos x \sin(x - 1)] y' + (x - 1) y = 0 \] in the interval \( \left[0, \frac{\pi}{2} \right] \) is equal to}{19}
\vspace{0.5cm}

\questiona{Let \( F \) be a field with 7\(^6\) elements and let \( K \) be a subfield of \( F \) with 49 elements. Then the dimension of \( F \) as a vector space over \( K \) is}{20}
\vspace{0.5cm}

\questiona{Let \( C([0, 1]) \) be the real vector space of all continuous real valued functions on [0, 1], and let \( T \) be the linear operator on \( C([0, 1]) \) given by \[ (Tf)(x) = \int_0^1 \sin(x + y)f(y) \, dy, \quad x \in [0, 1]. \] Then the dimension of the range space of \( T \) equals}{21}
\vspace{0.5cm}

\questiona{Let \( a \in (-1, 1) \) be such that the quadrature rule \[ \int_{-1}^{1} f(x)\, dx \approx f(-a) + f(a) \] is exact for all polynomials of degree less than or equal to 3. Then \( 3a^2 = \)}{22}
\vspace{0.5cm}

\questiona{Let \( X \) and \( Y \) have joint probability density function given by \[
f_{X,Y}(x, y) = \begin{cases}
2, & 0 \le x \le 1 - y,\, 0 \le y \le 1 \\
0, & \text{otherwise}
\end{cases}
\] If \( f_Y \) denotes the marginal probability density function of \( Y \), then \( f_Y(1/2) = \)}{23}
\vspace{0.5cm}

\questiona{Let the cumulative distribution function of the random variable \( X \) be given by
\[
F_X(x) =
\begin{cases}
0, & x < 0 \\
x, & 0 \le x < 1/2 \\
(1 + x)/2, & 1/2 \le x < 1 \\
1, & x \ge 1
\end{cases}
\]
Then \( \mathbb{P}(X = 1/2) = \)}{24}
\vspace{0.5cm}

\questiona{Let \( \{X_j\} \) be a sequence of independent Bernoulli random variables with \( \mathbb{P}(X_j = 1) = \frac{1}{4} \) and let \( Y_n = \frac{1}{n} \sum_{j=1}^{n} X_j^2 \). Then \( Y_n \) converges, in probability, to}{25}
\vspace{0.5cm}

\questionb{Let \( \Gamma \) be the circle given by \( z = 4e^{i\theta} \), where \( \theta \) varies from 0 to \( 2\pi \). Then \[ \oint_{\Gamma} \frac{e^z}{z^2 - 2z} dz = \]}{26}
\begin{enumerate}
    \item[(A)] \( 2\pi i (e^2 - 1) \)
    \item[(B)] \( \pi i (1 - e^2) \)
    \item[(C)] \( \pi i (e^2 - 1) \)
    \item[(D)] \( 2\pi i (1 - e^2) \)
\end{enumerate}
\vspace{0.5cm}

\questionb{The image of the half plane \( \text{Re}(z) + \text{Im}(z) > 0 \) under the map \( w = \frac{z - 1}{z + i} \) is given by}{27}
\begin{enumerate}
    \item[(A)] \( \text{Re}(w) > 0 \)
    \item[(B)] \( \text{Im}(w) > 0 \)
    \item[(C)] \( |w| > 1 \)
    \item[(D)] \( |w| < 1 \)
\end{enumerate}
\vspace{0.5cm}

\questionb{Let \( D \subset \mathbb{R}^2 \) denote the closed disc with center at the origin and radius 2. Then \[
\iint_D e^{-(x^2 + y^2)} \, dx\, dy =
\]}{28}
\begin{enumerate}
    \item[(A)] \( \pi (1 - e^{-4}) \)
    \item[(B)] \( \frac{\pi}{2} (1 - e^{-4}) \)
    \item[(C)] \( \pi (1 - e^{-2}) \)
    \item[(D)] \( \frac{\pi}{2} (1 - e^{-2}) \)
\end{enumerate}
\vspace{0.5cm}

\questionb{Consider the polynomial \( p(X) = X^4 + 4 \) in the ring \( \mathbb{Q}[X] \). Then}{29}
\begin{enumerate}
    \item[(A)] The set of zeros of \( p(X) \) in \( \mathbb{C} \) forms a group under multiplication
    \item[(B)] \( p(X) \) is reducible in the ring \( \mathbb{Q}[X] \)
    \item[(C)] The splitting field of \( p(X) \) has degree 3 over \( \mathbb{Q} \)
    \item[(D)] The splitting field of \( p(X) \) has degree 4 over \( \mathbb{Q} \)
\end{enumerate}
\vspace{0.5cm}

\questionb{Which one of the following statements is true?}{30}
\begin{enumerate}
    \item[(A)] Every group of order 12 has a non-trivial proper normal subgroup
    \item[(B)] Some group of order 12 does not have a non-trivial proper normal subgroup
    \item[(C)] Every group of order 12 has a subgroup of order 6
    \item[(D)] Every group of order 12 has an element of order 12
\end{enumerate}
\vspace{0.5cm}

\questionb{For an odd prime \( p \), consider the ring \( \mathbb{Z}[\sqrt{-p}] = \{a + b\sqrt{-p} : a, b \in \mathbb{Z}\} \subset \mathbb{C} \). Then the element 2 in \( \mathbb{Z}[\sqrt{-p}] \) is}{31}
\begin{enumerate}
    \item[(A)] a unit
    \item[(B)] a square
    \item[(C)] a prime
    \item[(D)] irreducible
\end{enumerate}
\vspace{0.5cm}

\questionb{Consider the following two statements: \\
P: The matrix \( \begin{bmatrix} 0 & 5 \\ 0 & 7 \end{bmatrix} \) has infinitely many LU factorizations, where \( L \) is lower triangular with each diagonal entry 1 and \( U \) is upper triangular. \\
Q: The matrix \( \begin{bmatrix} 0 & 0 \\ 2 & 5 \end{bmatrix} \) has no LU factorization, where \( L \) is lower triangular with each diagonal entry 1 and \( U \) is upper triangular. \\
Then which one of the following options is correct?}{32}
\begin{enumerate}
    \item[(A)] P is TRUE and Q is FALSE
    \item[(B)] Both P and Q are TRUE
    \item[(C)] P is FALSE and Q is TRUE
    \item[(D)] Both P and Q are FALSE
\end{enumerate}
\vspace{0.5cm}

\questionb{If the characteristic curves of the partial differential equation \( x u_{xx} + 2x^2 u_{xy} = u_x - 1 \) are \( \mu(x, y) = c_1 \) and \( \nu(x, y) = c_2 \), where \( c_1 \) and \( c_2 \) are constants, then}{33}
\begin{enumerate}
    \item[(A)] \( \mu(x, y) = x^2 - y, \nu(x, y) = y \)
    \item[(B)] \( \mu(x, y) = x^2 + y, \nu(x, y) = y \)
    \item[(C)] \( \mu(x, y) = x^2 + y, \nu(x, y) = x^2 \)
    \item[(D)] \( \mu(x, y) = x^2 - y, \nu(x, y) = x^2 \)
\end{enumerate}
\vspace{0.5cm}

\questionb{Let \( f : X \to Y \) be a continuous map from a Hausdorff topological space \( X \) to a metric space \( Y \). Consider the following two statements: \\
P: \( f \) is a closed map and the inverse image \( f^{-1}(y) = \{x \in X : f(x) = y\} \) is compact for each \( y \in Y \). \\
Q: For every compact subset \( K \subset Y \), the inverse image \( f^{-1}(K) \) is a compact subset of \( X \). \\
Which one of the following is true?}{34}
\begin{enumerate}
    \item[(A)] Q implies P but P does NOT imply Q
    \item[(B)] P implies Q but Q does NOT imply P
    \item[(C)] P and Q are equivalent
    \item[(D)] neither P implies Q nor Q implies P
\end{enumerate}
\vspace{0.5cm}

\questionb{Let \( X \) denote \( \mathbb{R}^2 \) endowed with the usual topology. Let \( Y \) denote \( \mathbb{R} \) endowed with the co-finite topology. If \( Z \) is the product topological space \( Y \times Y \), then}{35}
\begin{enumerate}
    \item[(A)] the topology of \( X \) is the same as the topology of \( Z \)
    \item[(B)] the topology of \( X \) is strictly coarser (weaker) than that of \( Z \)
    \item[(C)] the topology of \( Z \) is strictly coarser (weaker) than that of \( X \)
    \item[(D)] the topology of \( X \) cannot be compared with that of \( Z \)
\end{enumerate}
\vspace{0.5cm}

\questionb{Consider \( \mathbb{R}^n \) with the usual topology for \( n = 1, 2, 3 \). Each of the following options gives topological spaces \( X \) and \( Y \) with respective induced topologies. In which option is \( X \) homeomorphic to \( Y \)?}{36}
\begin{enumerate}
    \item[(A)] \( X = \{(x, y, z) \in \mathbb{R}^3 : x^2 + y^2 = 1\}, Y = \{(x, y, z) \in \mathbb{R}^3 : z = 0, x^2 + y^2 \neq 0\} \)
    \item[(B)] \( X = \{(x, y) \in \mathbb{R}^2 : y = \sin(1/x), 0 < x \le 1\} \cup \{(0, y) : -1 \le y \le 1\}, Y = [0, 1] \subset \mathbb{R} \)
    \item[(C)] \( X = \{(x, y) \in \mathbb{R}^2 : y = x \sin(1/x), 0 < x \le 1\}, Y = [0, 1] \subset \mathbb{R} \)
    \item[(D)] \( X = \{(x, y, z) \in \mathbb{R}^3 : x^2 + y^2 = 1\}, Y = \{(x, y, z) \in \mathbb{R}^3 : x^2 + y^2 = z^2 \neq 0\} \)
\end{enumerate}
\vspace{0.5cm}

\questionb{Let \( \{X_i\} \) be a sequence of independent Poisson(\( \lambda \)) variables and let \( W_n = \frac{1}{n} \sum_{i=1}^n X_i \). Then the limiting distribution of \( \sqrt{n}(W_n - \lambda) \) is the normal distribution with zero mean and variance given by}{37}
\begin{enumerate}
    \item[(A)] 1
    \item[(B)] \( \sqrt{\lambda} \)
    \item[(C)] \( \lambda \)
    \item[(D)] \( \lambda^2 \)
\end{enumerate}
\vspace{0.5cm}

\questionb{Let \( X_1, X_2, \ldots, X_n \) be independent and identically distributed random variables with probability density function given by \[
f_X(x; \theta) = \begin{cases}
\theta e^{-\theta(x - 1)}, & x \ge 1 \\
0, & \text{otherwise}
\end{cases}
\]
Also, let \( \bar{X} = \frac{1}{n} \sum_{i=1}^n X_i \). Then the maximum likelihood estimator of \( \theta \) is}{38}
\begin{enumerate}
    \item[(A)] \( \frac{1}{\bar{X}} \)
    \item[(B)] \( \frac{1}{\bar{X}} - 1 \)
    \item[(C)] \( \frac{1}{\bar{X} - 1} \)
    \item[(D)] \( \bar{X} \)
\end{enumerate}
\vspace{0.5cm}

\questionb{Consider the Linear Programming Problem (LPP):\\
Maximize \( \alpha x_1 + x_2 \)\\
Subject to: \\
\[
\begin{aligned}
2x_1 + x_2 &\le 6 \\
-x_1 + x_2 &\le 1 \\
x_1 + x_2 &\le 4 \\
x_1, x_2 &\ge 0
\end{aligned}
\]
If \( (3, 0) \) is the only optimal solution, then}{39}
\begin{enumerate}
    \item[(A)] \( \alpha < -2 \)
    \item[(B)] \( -2 < \alpha < 1 \)
    \item[(C)] \( 1 < \alpha < 2 \)
    \item[(D)] \( \alpha > 2 \)
\end{enumerate}
\vspace{0.5cm}

\questionb{Let \( M_2(\mathbb{R}) \) be the vector space of all \( 2 \times 2 \) real matrices over \( \mathbb{R} \). Define the linear transformation \( S : M_2(\mathbb{R}) \to M_2(\mathbb{R}) \) by \( S(X) = 2X + X^T \), where \( X^T \) denotes the transpose of the matrix \( X \). Then the trace of \( S \) equals}{40}
\vspace{0.5cm}

\questionb{Consider \( \mathbb{R}^3 \) with the usual inner product. If \( d \) is the distance from \( (1, 1, 1) \) to the subspace \( \text{span}\{(1, 1, 0), (0, 1, 1)\} \) of \( \mathbb{R}^3 \), then \( 3d^2 = \)}{41}
\vspace{0.5cm}

\questionb{Consider the matrix \( A = I_9 - 2u^T u \) with \( u = \frac{1}{3}[1, 1, 1, 1, 1, 1, 1, 1, 1] \), where \( I_9 \) is the \( 9 \times 9 \) identity matrix and \( u^T \) is the transpose of \( u \). If \( \lambda \) and \( \mu \) are two distinct eigenvalues of \( A \), then \( |\lambda - \mu| = \)}{42}
\vspace{0.5cm}

\questionb{Let \( f(z) = z^3 e^{z^2} \) for \( z \in \mathbb{C} \), and let \( \Gamma \) be the circle \( z = e^{i\theta} \), where \( \theta \) varies from 0 to \( 4\pi \). Then \[
\frac{1}{2\pi i} \oint_\Gamma \frac{f'(z)}{f(z)}\, dz = 
\]}{43}
\vspace{0.5cm}

\questionb{Let \( S \) be the surface of the solid \( V = \{(x, y, z) : 0 \le x \le 1, 0 \le y \le 2, 0 \le z \le 3\} \). Let \( \hat{n} \) denote the unit outward normal to \( S \) and let \( \vec{F}(x, y, z) = x\hat{i} + y\hat{j} + z\hat{k} \), \( (x, y, z) \in V \). Then the surface integral \( \iint_S \vec{F} \cdot \hat{n}\, dS \) equals}{44}
\vspace{0.5cm}

\questionb{Let \( A \) be a \( 3 \times 3 \) matrix with real entries. If three solutions of the linear system of differential equations \( \dot{x}(t) = Ax(t) \) are given by
\[
\begin{bmatrix} e^t - e^{2t} \\ -e^t + e^{2t} \\ e^t + e^{2t} \end{bmatrix},
\begin{bmatrix} -e^{2t} - e^{-t} \\ e^{2t} - e^{-t} \\ e^{2t} + e^{-t} \end{bmatrix},
\begin{bmatrix} e^{-t} + 2e^t \\ e^{-t} - 2e^t \\ -e^{-t} + 2e^t \end{bmatrix}
\]
then the sum of the diagonal entries of \( A \) is equal to}{45}
\vspace{0.5cm}

\questionb{If \( y_1(x) = e^{-x^2} \) is a solution of the differential equation
\[
x y'' + \alpha y' + \beta x^3 y = 0
\]
for some real numbers \( \alpha \) and \( \beta \), then \( \alpha\beta = \)}{46}
\vspace{0.5cm}

\questionb{Let \( L^2([0, 1]) \) be the Hilbert space of all real valued square integrable functions on [0, 1] with the usual inner product. Let \( \phi \) be the linear functional on \( L^2([0, 1]) \) defined by
\[
\phi(f) = \int_{1/4}^{3/4} \sqrt{2} f \, d\mu,
\]
where \( \mu \) denotes the Lebesgue measure on [0, 1]. Then \( \|\phi\| = \)}{47}
\vspace{0.5cm}

\questionb{Let \( U \) be an orthonormal set in a Hilbert space \( H \) and let \( x \in H \) be such that \( \|x\| = 2 \). Consider the set
\[
E = \left\{ u \in U : |\langle x, u \rangle| \geq \frac{1}{4} \right\}.
\]
Then the maximum possible number of elements in \( E \) is}{48}
\vspace{0.5cm}

\questionb{If \( p(x) = 2 - (x + 1) + x(x + 1) - \beta x(x + 1)(x - \alpha) \) interpolates the points \[
\begin{array}{c|cccc}
x & -1 & 0 & 1 & 2 \\
\hline
y & 2 & 1 & 2 & -7
\end{array}
\]
then \( \alpha + \beta = \)}{49}
\vspace{0.5cm}

\questionb{If \( \sin(\pi x) = a_0 + \sum_{n=1}^{\infty} a_n \cos(n \pi x) \) for \( 0 < x < 1 \), then \( (a_0 + a_1)\pi = \)}{50}
\vspace{0.5cm}

\questionb{For \( n = 1, 2, \ldots \), let \( f_n(x) = \frac{2n x^{n-1}}{1 + x} \), \( x \in [0, 1] \). Then
\[
\lim_{n \to \infty} \int_0^1 f_n(x)\, dx = 
\]}{51}
\vspace{0.5cm}

\questionb{Let \( X_1, X_2, X_3, X_4 \) be independent exponential random variables with mean 1, \( \frac{1}{2} \), \( \frac{1}{3} \), and \( \frac{1}{4} \), respectively. Then \( Y = \min(X_1, X_2, X_3, X_4) \) has exponential distribution with mean equal to}{52}
\vspace{0.5cm}

\questionb{Let \( X \) be the number of heads in 4 tosses of a fair coin by Person 1 and let \( Y \) be the number of heads in 4 tosses of a fair coin by Person 2. Assume that all the tosses are independent. Then the value of \( \mathbb{P}(X = Y) \), correct up to three decimal places, is}{53}
\vspace{0.5cm}

\questionb{Let \( X_1 \) and \( X_2 \) be independent geometric random variables with the same probability mass function given by
\[
\mathbb{P}(X = k) = p(1 - p)^{k - 1}, \quad k = 1, 2, \ldots.
\]
Then the value of \( \mathbb{P}(X_1 = 2 \mid X_1 + X_2 = 4) \), correct up to three decimal places, is}{54}
\vspace{0.5cm}

\questionb{A certain commodity is produced by the manufacturing plants \( P_1 \) and \( P_2 \) whose capacities are 6 and 5 units, respectively. The commodity is shipped to markets \( M_1, M_2, M_3 \), and \( M_4 \) whose requirements are 1, 2, 3, and 5 units, respectively. The transportation cost per unit from plant \( P_i \) to market \( M_j \) is as follows:

\[
\begin{array}{c|c|c|c|c|c}
 & M_1 & M_2 & M_3 & M_4 & \\
\hline
P_1 & 1 & 3 & 5 & 8 & 6\\
\hline
P_2 & 2 & 5 & 6 & 7 & 5\\
\hline
& 1 & 2 &3 &5
\end{array}
\]

Then the optimal cost of transportation is}{55}
\vspace{0.5cm}

\vspace{5cm}
\begin{center}
\textbf{END OF THE QUESTION PAPER} \\
\rule{\textwidth}{0.5pt}
\end{center}

\end{document}

