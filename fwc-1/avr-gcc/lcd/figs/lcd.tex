%\documentclass{standalone}
%\usepackage{tikz}
%\usepackage{amsmath,amssymb}
%\makeatletter
%\newsavebox\myboxA
%\newsavebox\myboxB
%\newlength\mylenA
%\newcommand*\xoverline[2][0.75]{%
%    \sbox{\myboxA}{$\m@th#2$}%
%    \setbox\myboxB\null% Phantom box
%    \ht\myboxB=\ht\myboxA%
%    \dp\myboxB=\dp\myboxA%
%    \wd\myboxB=#1\wd\myboxA% Scale phantom
%    \sbox\myboxB{$\m@th\overline{\copy\myboxB}$}%  Overlined phantom
%    \setlength\mylenA{\the\wd\myboxA}%   calc width diff
%    \addtolength\mylenA{-\the\wd\myboxB}%
%    \ifdim\wd\myboxB<\wd\myboxA%
%       \rlap{\hskip 0.5\mylenA\usebox\myboxB}{\usebox\myboxA}%
%    \else
%        \hskip -0.5\mylenA\rlap{\usebox\myboxA}{\hskip 0.5\mylenA\usebox\myboxB}%
%    \fi}
%\makeatother


%\begin{document}

\begin{tikzpicture}[scale=1,
     pin/.style={draw,rectangle,minimum width=2em,font=\small}
     ]
%   \clip (18,.5) rectangle (5,2);           
%Vertices of the main display rectangle
\def \xmin{0}
\def \xmax{17}
\def \ymin{0}
\def \ymax{6}

%Number of pins on a side
\def \n{8}
\def \k{1.6}

%Draw the display rectangle


%Define height of pins and their separation
\def \height{2}
\pgfmathsetmacro{\centx}{(\xmax+\xmin)/2}
\pgfmathsetmacro{\centy}{(\ymax+\ymin)/2}
\pgfmathsetmacro{\wid}{(\xmax-\xmin)/(\n-1)}


\draw ({\xmin-0.5*\wid},\ymin)rectangle ({\xmax+0.5*\wid},\ymax);
%\node (G) at (\centx,\centy) [draw,thick,minimum width=12.6cm,minimum height=4.8] {};


%Putting text 7447 at the centre
   \node at (\centx,\centy) {\textbf{\LARGE{LCD $ 16  \times 2$}}};
%   \node at (-1,0) {\textbf{$1$}};
%    \node at (14,0) {\textbf{$16$}};

      
\foreach [count=\i] \k in {}
   {
\pgfmathsetmacro{\j}{int(round(17-\i)}

            \node (\i) at ( {\xmin+(\i-1)*\wid},{\ymax+0.45*\wid}) {\LARGE \k} ;
   }

\foreach [count=\i] \k in {}
{
            \node (\i) at ( {\xmin+(\i-1)*\wid},{\ymin-0.45*\wid}) {\LARGE \k} ;              
 }
\node(G) at (1,0){};
\foreach \x in {1,...,16}
\node[above=2pt] at (\x,0) {\x};
% \node(G)[rectangle,fill=red!5,draw=red,text width=5cm]{};
%\node(Vcc)[below of=G]{Vcc};
\node(GND)[below of=G]{GND};
\node(Vcc)[right of=GND]{Vcc};
\node(Vee)[right of=Vcc]{Vee};
\node(RS)[right of=Vee]{RS};
\node(RW)[right of=RS]{RW};
\node(EN)[right of=RW]{EN};
\node(DB0)[right of=EN]{DB0};
\node(DB1)[right of=DB0]{DB1};
\node(DB2)[right of=DB1]{DB2};
\node(DB3)[right of=DB2]{DB3};
\node(DB4)[right of=DB3]{DB4};
\node(DB5)[right of=DB4]{DB5};
\node(DB6)[right of=DB5]{DB6};
\node(DB7)[right of=DB6]{DB7};
\node(LED+)[right of=DB7]{LED+};
\node(LED-)[right of=LED+]{LED-};
\draw[-](G.south)-|(GND.north);
\draw[-](G.south)-|(Vcc.north);
\draw[-](G.south)-|(Vee.north);
\draw[-](G.south)-|(RS.north);
\draw[-](G.south)-|(RW.north);
\draw[-](G.south)-|(EN.north);
\draw[-](G.south)-|(DB0.north);
\draw[-](G.south)-|(DB1.north);
\draw[-](G.south)-|(DB2.north);
\draw[-](G.south)-|(DB3.north);
\draw[-](G.south)-|(DB4.north);
\draw[-](G.south)-|(DB5.north);
\draw[-](G.south)-|(DB6.north);
\draw[-](G.south)-|(DB7.north);
\draw[-](G.south)-|(LED+.north);
\draw[-](G.south)-|(LED-.north);
\end{tikzpicture}
%\endpgfgraphicnamed
%\end{document}

%% \end{tikzpicture}
%\end{document}