\documentclass[journal,12pt,twocolumn]{IEEEtran}
%
\usepackage{setspace}
\usepackage{gensymb}
\usepackage{xcolor}
\usepackage{caption}
\usepackage[hyphens,spaces,obeyspaces]{url}
%\usepackage{subcaption}
%\doublespacing
\singlespacing

%\usepackage{graphicx}
%\usepackage{amssymb}
%\usepackage{relsize}
\usepackage[cmex10]{amsmath}
\usepackage{mathtools}
%\usepackage{amsthm}
%\interdisplaylinepenalty=2500
%\savesymbol{iint}
%\usepackage{txfonts}
%\restoresymbol{TXF}{iint}
%\usepackage{wasysym}
\usepackage{amsthm}
\usepackage{mathrsfs}
\usepackage{txfonts}
\usepackage{stfloats}
\usepackage{cite}
\usepackage{cases}
\usepackage{subfig}
%\usepackage{xtab}
\usepackage{longtable}
\usepackage{multirow}
%\usepackage{algorithm}
%\usepackage{algpseudocode}
\usepackage{enumerate}
\usepackage{mathtools}
\usepackage{eenrc}
%\usepackage[framemethod=tikz]{mdframed}
\usepackage[breaklinks]{hyperref}
%\usepackage{breakcites}
\usepackage{listings}
    \usepackage[latin1]{inputenc}                                 %%
    \usepackage{color}                                            %%
    \usepackage{array}                                            %%
    \usepackage{longtable}                                        %%
    \usepackage{calc}                                             %%
    \usepackage{multirow}                                         %%
    \usepackage{hhline}                                           %%
    \usepackage{ifthen}                                           %%
  %optionally (for landscape tables embedded in another document): %%
    \usepackage{lscape}     

\usepackage{tikz}
\usepackage{circuitikz}
\usepackage{karnaugh-map}
\usepackage{pgf}
\usepackage[hyphenbreaks]{breakurl}

%\usepackage{url}
%\def\UrlBreaks{\do\/\do-}





%\usepackage{stmaryrd}


%\usepackage{wasysym}
%\newcounter{MYtempeqncnt}
\DeclareMathOperator*{\Res}{Res}
%\renewcommand{\baselinestretch}{2}
\renewcommand\thesection{\arabic{section}}
\renewcommand\thesubsection{\thesection.\arabic{subsection}}
\renewcommand\thesubsubsection{\thesubsection.\arabic{subsubsection}}

\renewcommand\thesectiondis{\arabic{section}}
\renewcommand\thesubsectiondis{\thesectiondis.\arabic{subsection}}
\renewcommand\thesubsubsectiondis{\thesubsectiondis.\arabic{subsubsection}}

% correct bad hyphenation here
\hyphenation{op-tical net-works semi-conduc-tor}

%\lstset{
%language=C,
%frame=single, 
%breaklines=true
%}

%\lstset{
	%%basicstyle=\small\ttfamily\bfseries,
	%%numberstyle=\small\ttfamily,
	%language=Octave,
	%backgroundcolor=\color{white},
	%%frame=single,
	%%keywordstyle=\bfseries,
	%%breaklines=true,
	%%showstringspaces=false,
	%%xleftmargin=-10mm,
	%%aboveskip=-1mm,
	%%belowskip=0mm
%}

%\surroundwithmdframed[width=\columnwidth]{lstlisting}
\def\inputGnumericTable{}                                 %%
\lstset{
%language=C,
frame=single, 
breaklines=true,
columns=fullflexible
}
 

\begin{document}
%

\theoremstyle{definition}
\newtheorem{theorem}{Theorem}[section]
\newtheorem{problem}{Problem}
\newtheorem{proposition}{Proposition}[section]
\newtheorem{lemma}{Lemma}[section]
\newtheorem{corollary}[theorem]{Corollary}
\newtheorem{example}{Example}[section]
\newtheorem{definition}{Definition}[section]
%\newtheorem{algorithm}{Algorithm}[section]
%\newtheorem{cor}{Corollary}
\newcommand{\BEQA}{\begin{eqnarray}}
\newcommand{\EEQA}{\end{eqnarray}}
\newcommand{\define}{\stackrel{\triangle}{=}}

\bibliographystyle{IEEEtran}
%\bibliographystyle{ieeetr}

\providecommand{\nCr}[2]{\,^{#1}C_{#2}} % nCr
\providecommand{\nPr}[2]{\,^{#1}P_{#2}} % nPr
\providecommand{\mbf}{\mathbf}
\providecommand{\pr}[1]{\ensuremath{\Pr\left(#1\right)}}
\providecommand{\qfunc}[1]{\ensuremath{Q\left(#1\right)}}
\providecommand{\sbrak}[1]{\ensuremath{{}\left[#1\right]}}
\providecommand{\lsbrak}[1]{\ensuremath{{}\left[#1\right.}}
\providecommand{\rsbrak}[1]{\ensuremath{{}\left.#1\right]}}
\providecommand{\brak}[1]{\ensuremath{\left(#1\right)}}
\providecommand{\lbrak}[1]{\ensuremath{\left(#1\right.}}
\providecommand{\rbrak}[1]{\ensuremath{\left.#1\right)}}
\providecommand{\cbrak}[1]{\ensuremath{\left\{#1\right\}}}
\providecommand{\lcbrak}[1]{\ensuremath{\left\{#1\right.}}
\providecommand{\rcbrak}[1]{\ensuremath{\left.#1\right\}}}
\providecommand{\ceil}[1]{\left \lceil #1 \right \rceil }
\theoremstyle{remark}
\newtheorem{rem}{Remark}
\newcommand{\sgn}{\mathop{\mathrm{sgn}}}
\providecommand{\abs}[1]{\left\vert#1\right\vert}
\providecommand{\res}[1]{\Res\displaylimits_{#1}} 
\providecommand{\norm}[1]{\lVert#1\rVert}
\providecommand{\mtx}[1]{\mathbf{#1}}
\providecommand{\mean}[1]{E\left[ #1 \right]}
\providecommand{\fourier}{\overset{\mathcal{F}}{ \rightleftharpoons}}
%\providecommand{\hilbert}{\overset{\mathcal{H}}{ \rightleftharpoons}}
\providecommand{\system}{\overset{\mathcal{H}}{ \longleftrightarrow}}
	%\newcommand{\solution}[2]{\textbf{Solution:}{#1}}
\newcommand{\solution}{\noindent \textbf{Solution: }}
\providecommand{\dec}[2]{\ensuremath{\overset{#1}{\underset{#2}{\gtrless}}}}
%\numberwithin{equation}{subsection}
\numberwithin{equation}{section}
%\numberwithin{problem}{subsection}
%\numberwithin{definition}{subsection}
\makeatletter
\@addtoreset{figure}{problem}
\makeatother

\let\StandardTheFigure\thefigure
%\renewcommand{\thefigure}{\theproblem.\arabic{figure}}
\renewcommand{\thefigure}{\theproblem}


%\numberwithin{figure}{subsection}

%\numberwithin{equation}{subsection}
%\numberwithin{equation}{section}
%%\numberwithin{equation}{problem}
%%\numberwithin{problem}{subsection}
\numberwithin{problem}{section}
%%\numberwithin{definition}{subsection}
%\makeatletter
%\@addtoreset{figure}{problem}
%\makeatother
\makeatletter
\@addtoreset{table}{problem}
\makeatother

\let\StandardTheFigure\thefigure
\let\StandardTheTable\thetable
%%\renewcommand{\thefigure}{\theproblem.\arabic{figure}}
%\renewcommand{\thefigure}{\theproblem}
\renewcommand{\thetable}{\theproblem}
%%\numberwithin{figure}{section}

%%\numberwithin{figure}{subsection}

\vspace{3cm}

\title{ 
	\logo{
Finite State Machine
	}
}



% paper title
% can use linebreaks \\ within to get better formatting as desired
%\title{Matrix Analysis through Octave}
%
%
% author names and IEEE memberships
% note positions of commas and nonbreaking spaces ( ~ ) LaTeX will not break
% a structure at a ~ so this keeps an author's name from being broken across
% two lines.
% use \thanks{} to gain access to the first footnote area
% a separate \thanks must be used for each paragraph as LaTeX2e's \thanks
% was not built to handle multiple paragraphs
%

\author{G V V Sharma$^{*}$% <-this % stops a space
\thanks{*The author is with the Department
of Electrical Engineering, Indian Institute of Technology, Hyderabad
502285 India e-mail:  gadepall@iith.ac.in. All content in this manual is released under GNU GPL.  Free and open source.}% <-this % stops a space
%\thanks{J. Doe and J. Doe are with Anonymous University.}% <-this % stops a space
%\thanks{Manuscript received April 19, 2005; revised January 11, 2007.}}
}
% note the % following the last \IEEEmembership and also \thanks - 
% these prevent an unwanted space from occurring between the last author name
% and the end of the author line. i.e., if you had this:
% 
% \author{....lastname \thanks{...} \thanks{...} }
%                     ^------------^------------^----Do not want these spaces!
%
% a space would be appended to the last name and could cause every name on that
% line to be shifted left slightly. This is one of those "LaTeX things". For
% instance, "\textbf{A} \textbf{B}" will typeset as "A B" not "AB". To get
% "AB" then you have to do: "\textbf{A}\textbf{B}"
% \thanks is no different in this regard, so shield the last } of each \thanks
% that ends a line with a % and do not let a space in before the next \thanks.
% Spaces after \IEEEmembership other than the last one are OK (and needed) as
% you are supposed to have spaces between the names. For what it is worth,
% this is a minor point as most people would not even notice if the said evil
% space somehow managed to creep in.



% The paper headers
%\markboth{Journal of \LaTeX\ Class Files,~Vol.~6, No.~1, January~2007}%
%{Shell \MakeLowercase{\textit{et al.}}: Bare Demo of IEEEtran.cls for Journals}
% The only time the second header will appear is for the odd numbered pages
% after the title page when using the twoside option.
% 
% *** Note that you probably will NOT want to include the author's ***
% *** name in the headers of peer review papers.                   ***
% You can use \ifCLASSOPTIONpeerreview for conditional compilation here if
% you desire.




% If you want to put a publisher's ID mark on the page you can do it like
% this:
%\IEEEpubid{0000--0000/00\$00.00~\copyright~2007 IEEE}
% Remember, if you use this you must call \IEEEpubidadjcol in the second
% column for its text to clear the IEEEpubid mark.



% make the title area
\maketitle

\tableofcontents

\bigskip

\renewcommand{\thefigure}{\theenumi}
\renewcommand{\thetable}{\theenumi}


\begin{abstract}
%\boldmath
This manual explains state machines by deconstructing a decade counter.
\end{abstract}

\section{The Decade Counter}
The block diagram of a decade counter (repeatedly counts up from 0 to 9)
is available in Fig. \ref{fig:dec_counter}.  The {\em incrementing } decoder
and {\em display} decoder are part of {\em combinational} logic, while
the {\em delay} is part of {\em sequential} logic.
\begin{figure}[!h]
\resizebox {\columnwidth} {!} {
%\documentclass{article}

%\usepackage[latin1]{inputenc}
%\usepackage{tikz}
%\usetikzlibrary{shapes,arrows}

%%%%<
%\usepackage{verbatim}
%\usepackage[active,tightpage]{preview}
%\PreviewEnvironment{tikzpicture}
%\setlength\PreviewBorder{5pt}%
%%%%>

%\begin{comment}
%:Title: Simple flow chart
%:Tags: Diagrams

%With PGF/TikZ you can draw flow charts with relative ease. This flow chart from [1]_
%outlines an algorithm for identifying the parameters of an autonomous underwater vehicle model. 

%Note that relative node
%placement has been used to avoid placing nodes explicitly. This feature was
%introduced in PGF/TikZ >= 1.09.

%.. [1] Bossley, K.; Brown, M. & Harris, C. Neurofuzzy identification of an autonomous underwater vehicle `International Journal of Systems Science`, 1999, 30, 901-913 


%\end{comment}


%\begin{document}
%\pagestyle{empty}


% Define block styles
\tikzstyle{decision} = [diamond, draw, fill=blue!20, 
    text width=4.5em, text badly centered, node distance=3cm, inner sep=0pt]
%\tikzstyle{block} = [rectangle, draw, fill=blue!20, 
%    text width=5em, text centered, rounded corners, minimum height=4em]
\tikzstyle{block} = [rectangle, draw, 
    text width=5em, text centered, rounded corners, minimum height=4em]

\tikzstyle{line} = [draw, -latex']
\tikzstyle{cloud} = [draw, ellipse,fill=red!20, node distance=3cm,
    minimum height=2em]
    
\begin{tikzpicture}[node distance = 3cm, auto]
    % Place nodes
    \node [block] (init) {Incrementing Decoder};
%    \node [cloud, left of=init] (expert) {expert};
%    \node [cloud, right of=init] (system) {system};
    \node [block, below of=init, node distance = 4cm] (identify) {Display Decoder};
    \node [block, below of=identify ] (evaluate) {Seven-Segment Display};
%    \node [block, right of=identify, node distance = 4cm] (delay) {Delay};
     %\node [block, (4,-3)] (q1) {Delay};
	\node at (4,-2)[block] (delay) {Delay};
\begin{scope}[->,>=latex]
    \foreach \i in {-3,-1,1,3}
    { 
%      \draw[->] ([yshift=\i * 0.2 cm]identify.east) -- ([yshift=\i * 0.2 cm]delay.west) ;
      \draw[->] ([xshift=\i * 0.2 cm]delay.north) |- ([yshift=\i * 0.2 cm]init.east) ;
      \draw[->] ([xshift=\i * 0.2 cm]init.south) -- ([xshift=\i * 0.2 cm]identify.north) ;
       \draw node at (\i * 0.2,-2+\i * 0.2) { \textbullet} ;
       \draw[->] (\i * 0.2,-2+\i * 0.2) -- ([yshift=\i * 0.2 cm]delay.west) ;
      
    }
\foreach \i in {-3,...,3}
    { 
      \draw[->] ([xshift=\i * 0.35 cm]identify.south) -- ([xshift=\i * 0.35 cm]evaluate.north) ;
    }
\foreach [count=\i] \j in {a,b,...,g}{
            \node (\i) at ( 1.6-\i * 0.35, -5.5) {\j} ;
            }
\foreach [count=\i] \j in {A,B,C,D}{
            \node (\i) at ( 0.8-\i * 0.4, -1.0-\i*0.4) {\j} ;
            }

\foreach [count=\i] \j in {W,X,Y,Z}{
            \node (\i) at ( 1.6, 1.2-\i*0.4) {\j} ;
            }
    
\end{scope}

 %   \node [block, left of=evaluate, node distance=3cm] (update) {update model};
  %  \node [decision, below of=evaluate] (decide) {is best candidate better?};
%    \node [block, below of=decide, node distance=3cm] (stop) {stop};
    % Draw edges
%    \path [line] (init) -- (identify);
    \path [line] (identify) -- (evaluate);
%    \path [line] (evaluate) -- (decide);
  %  \path [line] (decide) -| node [near start] {yes} (update);
   % \path [line] (update) |- (identify);
 %   \path [line] (decide) -- node {no}(stop);
%    \path [line,dashed] (expert) -- (init);
%    \path [line,dashed] (system) -- (init);
%    \path [line,dashed] (system) |- (evaluate);
\end{tikzpicture}
%}

%\end{document}

}
\caption{The decade counter}
\label{fig:dec_counter}
\end{figure}
%
\section{Finite State Machine}
%
\begin{enumerate}[1.]

\item Fig. \ref{fig:fsm_counter} shows a {\em finite state machine} (FSM) diagram for the decade counter in Fig. \ref{fig:dec_counter}.  $s_0$ is the state when the input to the incrementing decoder is 0.  The {\em state transition table} for the FSM is Table 0 in \cite{gvv_kmap} where the present state is denoted by the variables $W,X,Y,Z$ and the next state by $A,B,C,D$.  
\begin{figure}[!h]
\centering
%\resizebox {\columnwidth} {!} {
\usetikzlibrary{arrows,automata, positioning, calc}
%\usetikzlibrary{arrows,automata, calc}
%\begin{tikzpicture}[->,shorten >=1pt,node distance=2cm,on grid,auto] 
\begin{tikzpicture}[->,auto] 
   \node[ ] (s_00)   {}; 
   \foreach \i [count=\ni from 1] in {36,72,...,324}
%       \node[state] (s_\ni) [above right = {2*sin(\i)} and {2*(cos(\i)} of s_00]  {\ni};
        \node[state] (s_\ni) [above right = {2*sin(\i)} and {2*(cos(\i)} of s_00]  {$s_{\ni}$};        
        
        \node[state,initial] (s_0) [above right = {0} and {2} of s_00]  {$s_0$};     

   \foreach \i  [count=\j from 1] in {0,1,...,8}
		\path	(s_\i) edge [bend right]  (s_\j) ;

		\path	(s_9) edge [bend right]  (s_0) ;		
           
\end{tikzpicture}

%}
\caption{FSM for the decade counter.}
\label{fig:fsm_counter}
\end{figure}
\item The FSM implementation is available in Fig. \ref{fig:dff}.  The {\em flip-flops} hold the input for the time that is given by the {\em clock}.  This is nothing but the implementation of the {\em Delay} block in Fig. \ref{fig:dec_counter}.
%
\begin{figure}[!h]
\resizebox {\columnwidth} {!} {
%\documentclass{standalone}
%\usepackage{pgf,tikz}
%\usetikzlibrary{calc,arrows}
%\usepackage{amsmath}

\makeatletter

% Data Flip Flip (DFF) shape
\pgfdeclareshape{dff}{
  % The 'minimum width' and 'minimum height' keys, not the content, determine
  % the size
  \savedanchor\northeast{%
    \pgfmathsetlength\pgf@x{\pgfshapeminwidth}%
    \pgfmathsetlength\pgf@y{\pgfshapeminheight}%
    \pgf@x=0.5\pgf@x
    \pgf@y=0.5\pgf@y
  }
  % This is redundant, but makes some things easier:
  \savedanchor\southwest{%
    \pgfmathsetlength\pgf@x{\pgfshapeminwidth}%
    \pgfmathsetlength\pgf@y{\pgfshapeminheight}%
    \pgf@x=-0.5\pgf@x
    \pgf@y=-0.5\pgf@y
  }
  % Inherit from rectangle
  \inheritanchorborder[from=rectangle]

  % Define same anchor a normal rectangle has
  \anchor{center}{\pgfpointorigin}
  \anchor{north}{\northeast \pgf@x=0pt}
  \anchor{east}{\northeast \pgf@y=0pt}
  \anchor{south}{\southwest \pgf@x=0pt}
  \anchor{west}{\southwest \pgf@y=0pt}
  \anchor{north east}{\northeast}
  \anchor{north west}{\northeast \pgf@x=-\pgf@x}
  \anchor{south west}{\southwest}
  \anchor{south east}{\southwest \pgf@x=-\pgf@x}
  \anchor{text}{
    \pgfpointorigin
    \advance\pgf@x by -.5\wd\pgfnodeparttextbox%
    \advance\pgf@y by -.5\ht\pgfnodeparttextbox%
    \advance\pgf@y by +.5\dp\pgfnodeparttextbox%
  }

  % Define anchors for signal ports
  \anchor{D}{
    \pgf@process{\northeast}%
    \pgf@x=-1\pgf@x%
    \pgf@y=.5\pgf@y%
  }
  \anchor{CLK}{
    \pgf@process{\northeast}%
    \pgf@x=-1\pgf@x%
    \pgf@y=-.66666\pgf@y%
  }
  \anchor{CE}{
    \pgf@process{\northeast}%
    \pgf@x=-1\pgf@x%
    \pgf@y=-0.33333\pgf@y%
  }
  \anchor{Q}{
    \pgf@process{\northeast}%
    \pgf@y=.5\pgf@y%
  }
  \anchor{Qn}{
    \pgf@process{\northeast}%
    \pgf@y=-.5\pgf@y%
  }
  \anchor{R}{
    \pgf@process{\northeast}%
    \pgf@x=0pt%
  }
  \anchor{S}{
    \pgf@process{\northeast}%
    \pgf@x=0pt%
    \pgf@y=-\pgf@y%
  }
  % Draw the rectangle box and the port labels
  \backgroundpath{
    % Rectangle box
    \pgfpathrectanglecorners{\southwest}{\northeast}
    % Angle (>) for clock input
    \pgf@anchor@dff@CLK
    \pgf@xa=\pgf@x \pgf@ya=\pgf@y
    \pgf@xb=\pgf@x \pgf@yb=\pgf@y
    \pgf@xc=\pgf@x \pgf@yc=\pgf@y
    \pgfmathsetlength\pgf@x{1.6ex} % size depends on font size
    \advance\pgf@ya by \pgf@x
    \advance\pgf@xb by \pgf@x
    \advance\pgf@yc by -\pgf@x
    \pgfpathmoveto{\pgfpoint{\pgf@xa}{\pgf@ya}}
    \pgfpathlineto{\pgfpoint{\pgf@xb}{\pgf@yb}}
    \pgfpathlineto{\pgfpoint{\pgf@xc}{\pgf@yc}}
    \pgfclosepath

    % Draw port labels
    \begingroup
    \tikzset{flip flop/port labels} % Use font from this style
    \tikz@textfont

    \pgf@anchor@dff@D
    \pgftext[left,base,at={\pgfpoint{\pgf@x}{\pgf@y}},x=\pgfshapeinnerxsep]{\raisebox{-0.75ex}{D}}

    \pgf@anchor@dff@CE
    \pgftext[left,base,at={\pgfpoint{\pgf@x}{\pgf@y}},x=\pgfshapeinnerxsep]{\raisebox{-0.75ex}{CE}}

    \pgf@anchor@dff@Q
    \pgftext[right,base,at={\pgfpoint{\pgf@x}{\pgf@y}},x=-\pgfshapeinnerxsep]{\raisebox{-.75ex}{Q}}

    \pgf@anchor@dff@Qn
    \pgftext[right,base,at={\pgfpoint{\pgf@x}{\pgf@y}},x=-\pgfshapeinnerxsep]{\raisebox{-.75ex}{$\overline{\mbox{Q}}$}}

    \pgf@anchor@dff@R
    \pgftext[top,at={\pgfpoint{\pgf@x}{\pgf@y}},y=-\pgfshapeinnerysep]{R}

    \pgf@anchor@dff@S
    \pgftext[bottom,at={\pgfpoint{\pgf@x}{\pgf@y}},y=\pgfshapeinnerysep]{S}
    \endgroup
  }
}

% Key to add font macros to the current font
\tikzset{add font/.code={\expandafter\def\expandafter\tikz@textfont\expandafter{\tikz@textfont#1}}} 

% Define default style for this node
\tikzset{flip flop/port labels/.style={font=\sffamily\scriptsize}}
\tikzset{every dff node/.style={draw,minimum width=2cm,minimum 
height=2.828427125cm,very thick,inner sep=1mm,outer sep=0pt,cap=round,add 
font=\sffamily}}
\tikzstyle{block} = [rectangle, draw, 
    text width=5em, text centered, rounded corners, minimum height=4em]


%\makeatother

%\begin{document}

\begin{tikzpicture}[font=\sffamily,>=triangle 45]
  \def\N{3}  % Number of Flip-Flops minus one

  % Place FFs
    \foreach \i [count=\m from 0] in {A,B,C,D}  
       \node [shape=dff] (DFF\m) at ($ 3*(0,\m) $) {\i};
%  \foreach \m in {0,...,\N}
%    \node [shape=dff] (DFF\m) at ($ 3*(0,\m) $) {Bit \#\m};

%  \def\N{7}  % Number of Flip-Flops minus one
%
%  % Place FFs
%  \foreach \m in {0,...,\N}
%    \node [shape=dff] (DFF\m) at ($ 3*(\m,0) $) {Bit \#\m};
%
%  % Connect FFs (Q1 with D1, etc.)
%  \def\p{0}  % Used to save the previous number
%  \foreach \m in {1,...,\N} { % Note that it starts with 1, not 0
%    \draw [->] (DFF\p.Q) -- (DFF\m.D);
%    \global\let\p\m
%  }
%
  % Connect and label data in- and output port
%  \draw [<-] (DFF0.D) -- +(-1,0) node [anchor=east] {input} ;
%  \draw [->] (DFF\N.Q) -- +(1,0) node [anchor=west] {output};
%
%  % 'Reset' port label
%  \path (DFF0) +(-2cm,+2cm) coordinate (temp)
%    node [anchor=east] {reset};
%  % Connect resets
%  \foreach \m in {0,...,\N}
%    \draw [->] (temp) -| (DFF\m.R);
%
%  % 'Set' port label
%  \path (DFF0) +(-2cm,-2cm) coordinate (temp)
%    node [anchor=east] {set};
%  % Connect sets
%  \foreach \m in {0,...,\N}
%    \draw [->] (temp) -| (DFF\m.S);
%
  % Clock port label
  \path (DFF0) +(-2cm,-2.5cm) coordinate (temp)
    node [anchor=east] {clock};
  \foreach \m in {0,...,\N}
    \draw [->] (temp) -| ($ (DFF\m.CLK) + (-5mm,0) $) --(DFF\m.CLK);
%
%  % Clock port label
%  \path (DFF0) +(-2cm,-3cm) coordinate (temp)
%    node [anchor=east] {clock enable};
%  \foreach \m in {0,...,\N}
%    \draw [->] (temp) -| ($ (DFF\m.CE) + (-7.5mm,0) $) --(DFF\m.CE);
	\node at (-4,12)[block] (init) {Incrementing Decoder};    
%	\node at (4,-2)[block] (delay) {Delay};	
    \foreach \i [count=\ni from 0] in {-3,-1,1,3}
    { 
      \draw  (DFF\ni.Q) -- +({\ni+1},0) node (output\ni){\textbullet};
      \draw[->] (output\ni) |- ([yshift=\i * 0.2 cm]init.east) ;
      \draw[->] ([xshift=\i * 0.2 cm]init.south) |- (DFF\ni.D);
%       -- ([xshift=\i * 0.2 cm]identify.north) ;      


      
    }
    \foreach [count=\i] \j in {W,X,Y,Z}{
%            \node (\i) at ( 1.6, 1.2-\i*0.4) {\j} ;
%            \node (\i) at ($([yshift=\i * 0.4 cm]init.east)-(0,1)$) {\j} ;
            \node (\i) at ($([yshift=\i * 0.4 cm]init.east)-(-0.2,0.8)$) {\scriptsize \j} ;
            }
\foreach [count=\i] \j in {A,B,C,D}{
            \node (\i) at ($([xshift=\i * 0.4 cm]init.south)-(0.9,0.2)$) {\scriptsize \j} ;
            }

	
\end{tikzpicture}

%\end{document}

}
\caption{Decade counter FSM implementation using D-Flip Flops.}
\label{fig:dff}
\end{figure}
%
\item The hardware cost of the system is given by
\begin{equation}
\text{No. of D Flip-Flops} = \ceil{\log_{2}\brak{\text{No. of States}}}
\end{equation}
For the FSM in Fig. \ref{fig:fsm_counter}, the number of states is 9, hence the number flipflops required = 4.  
\item Draw the state transition diagram for 
a decade down counter (counts from 9 to 0 repeatedly) using an FSM.  
\item Write the state transition table for the down counter.
\item Obtain the state transition equations with and without don't cares.
\item Verify your design using an arduino.
\item Repeat the above exercises by designing a circuit that can detect 3 consecutive 1s in a bitstream. 
\end{enumerate}
\bibliography{IEEEabrv,gvv_kmap_fsm}
\end{document}


